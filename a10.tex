\documentclass[12pt]{article}

\usepackage{amsmath,amssymb,latexsym,enumitem,xcolor,array}
\usepackage{algorithmicx,algorithm,algpseudocode,graphicx}

\setlength{\oddsidemargin}{-0.5in}
\setlength{\evensidemargin}{-0.5in}
\setlength{\textwidth}{7.5in}
\setlength{\topmargin}{0in}
\setlength{\textheight}{8.5in}

\newcommand{\answer}[1]{\underline{\hspace*{#1mm}}}
\newcommand{\up}[1]{\vspace*{-#1mm}}
\renewcommand{\arraystretch}{1}
\newcommand{\qed}{\hfill$\square$}
\newcommand{\assignNum}{10}
\newcommand{\numPoints}{40 (+ 5 bonus)}
\newcommand{\dueDate}{Monday 11/19/2018}
\setlength{\parindent}{0mm}
\newcommand{\header}{
\noindent{\textit{CS 321 - Algorithms - Fall 2018\hfill \yourName}}
\begin{center}
  \textbf{\Large Assignment \assignNum}\\\colorbox{red}{\Large\bf
     \textcolor{white}{Due BEFORE 8:00AM on  \dueDate}}\\\bigskip
  On time /  20\% off / no credit\\\bigskip
  \textbf{Total points: \numPoints}
  \bigskip

  \hrule\medskip You are allowed to work in teams of two or three on
  this assignment. Make sure to include all names above but submit
  only ONE zip  file to D2L.
  \medskip \hrule \end{center}

\medskip This assignment will help you develop your algorithm design
and analysis skills with a focus on dynamic-programming and greedy
algorithms.  As always, you must write up your solutions to this
assignment IN THIS FILE using \LaTeX\ by filling in (or replacing) all
of the boxes below.
% If your submitted \texttt{.tex}
% file does not compile, then you will receive 0 points.\medskip
You should NOT add any \LaTeX\ packages to this \texttt{.tex}
file. You will submit your code for Problem 2 in the file {\tt A10.java}.
\medskip

\medskip
\textbf{Submission procedure:}
\begin{enumerate}[itemsep=-2mm]

\item Complete this file, called \texttt{a\assignNum .tex}, with your
  full name(s) and answers typed up below.

\item Create a directory called \texttt{a\assignNum} and copy exactly three
files into this directory, namely:\up{4}

  \begin{itemize}[itemsep=-1mm]
  \item \texttt{a\assignNum .tex} (this file with all of your answers added)
  \item \texttt{a\assignNum .pdf} (the compiled version of the file above)
  \item \texttt{A10.java} (the file containing your answer to Problem 2)
  \end{itemize}

\item Zip up this directory to yield a file called \texttt{a\assignNum .zip}

\item Submit this zip file to the D2L dropbox for A\assignNum\  before the
deadline above.

\item Submit a STAPLED, SINGLE-SIDED, hard copy of your \texttt{a\assignNum
.pdf} file BEFORE the beginning of class on the due date above.
\end{enumerate}
}% end of header command

%************************************
% No need to modify anything above this line
% but DO fill this in!

\newcommand{\yourName}{\fbox{Your names go here}}

%************************************g


\begin{document}
\header

\textbf{Problem statements}
\begin{enumerate}

  %%%%%%%%%%%%%%%%%%%%%%%%%%%%%%%%%%%%%%%%%%%%%%%%%%%%%%%%%%%%%%%%%%%%%%%%%%%
   \item                            % Part 1
  %%%%%%%%%%%%%%%%%%%%%%%%%%%%%%%%%%%%%%%%%%%%%%%%%%%%%%%%%%%%%%%%%%%%%%%%%%%

     {\bf \color{red} (10 points) Consider the algorithm on Slide
       27-8. We proved that this greedy algorithm always returns an
       optimal solution to the activity-selection problem. For each
       one of the two parts below, state whether or not this algorithm
       always returns an optimal solution after line 2 has been
       replaced with a new type of greedy choice.

       For each part, you must justify your answer convincingly and AS
       CONCISELY AS POSSIBLE. Make sure that all answers are explicit with
       no work to be done by the grader except reading your justification.

       a. Proposed replacement for Line 2:

       {\tt Sort $A$ in order of
       non-decreasing duration $f_i - s_i$ to yield a sequence $A'$}\bigskip

       Optimal? \fbox{\color{black}{True /  False  (delete the wrong answer)}}\bigskip

       Justification:\medskip

      \fbox{\color{black}{Your justification goes here}}\bigskip

      b. Replace Line 2 with:


      $A' \leftarrow A$

      and replace Line 5 with:

      {\tt $a_k \leftarrow$ the activity in $A'$ that overlaps the smallest number of activities in $A'$ }\bigskip

       Optimal? \fbox{\color{black}{True /  False  (delete the wrong answer)}}\bigskip

       Justification:\medskip

      \fbox{\color{black}{Your justification goes here}}\bigskip


     }

  %%%%%%%%%%%%%%%%%%%%%%%%%%%%%%%%%%%%%%%%%%%%%%%%%%%%%%%%%%%%%%%%%%%%%%%%%%%
   \item                            % Part 2
  %%%%%%%%%%%%%%%%%%%%%%%%%%%%%%%%%%%%%%%%%%%%%%%%%%%%%%%%%%%%%%%%%%%%%%%%%%%


     {\bf \color{red} (30 + 5 points) You have been contacted by the
    International Ski Federation to help a future World Cup hosting
    country design their Super Giant Slalom course. This is an

    \begin{minipage}{0.55\linewidth}
     alpine skiing event in which both speed and the number of turns
     matter. The selected site is a mountain side with a 450-meter
     vertical drop and many possible gate locations represented by
     vertices in the graph on the right, in which vertex 1 rand 9 represent the
     top and bottom of the course, respectively. The number on each
     edge encodes the overall quality of each gate-to-gate segment in
     terms of vertical drop and other speed-enhancing factors, like
     the flatness of the ground, the wind protection provided by nearby
     trees, etc.\\

     While there is a minimum number of turns (and therefore of gates) that must
     be included in a super-G course, this number cannot be too high in
     order to increase the maximum speed that
     \end{minipage}\hfill
     \begin{minipage}{0.4\linewidth}
     \centerline{\includegraphics[width=\linewidth]{./graph.eps}} \end{minipage}

    skiers will reach. Your task is to find a maximum-quality path from the top to the
    bottom vertex. You will use a dynamic-programming approach to
    solve this problem. Of course, as an experienced computer
    scientist, you will test your algorithm and its implementation on
    many mountain sides whose characteristics are encoded in an input
    file with the following structure:

    \begin{itemize}
    \item The first number in the file is the minimum number of turns
      that the final course must have. Note that the top and bottom
      vertices can never count as a turn.
    \item The second number in the file is the penalty that the
      selected course will incur for each extra turn over the minimum
      number of required turns. This is a positive number that must be
      subtracted from the total path  quality for each extra turn.
    \item The third number in the file is the size of the mountain
      side defined as the exact number of course segments from the top
      of the mountain to the leftmost (or rightmost) gate
      position. This value is equal to 2 in the example depicted
      above, since that is the length of the direct path between
      vertices 1 and 3 (or between vertices 1 and 7).
    \item The rest of the numbers in the file are the numbers encoding
      the quality of each gate-to-gate segment. These numbers are
      always listed in the same order. Picture the mountain side
      rotated by 45 degrees counter-clockwise around vertex 1. Then
      all of the horizontal edges appear first, listed from left to
      right and then top to bottom, followed by the vertical edges,
      also listed from left to right and then top to bottom.
    \end{itemize}

    With these formatting constraints, here are the contents of the
    input file corresponding to the mountain side depicted above,
    assuming that at least two turns are required and that each extra
    turn incurs a 3-point penalty:

\begin{verbatim}
2 3 2 10 15 10 8 4 8 5 17 8 3 10 7
\end{verbatim}

Note that consecutive numbers are separated by one space character and
that the file always contains a single line of text whose length is
fully determined by the third number.

There are three versions of this problem for you to choose from:
\begin{enumerate}
\item \underline{Basic version:} You must find the  quality of a maximum-quality
  path while NOT enforcing the minimum required number of turns. In this
  version of the problem, each and every turn incurs the penalty
  specified by the second number in the file. You will ignore the
  first number. \underline{If you choose this version, your maximum possible
    score}
  \underline{on this problem will be 30 out of 30.}
\item \underline{Simplified version:} You must find the quality of a
  maximum-quality path without taking turns into account at all. You
  will ignore the first two numbers in the file. \underline{If you
    choose this version, your maximum possible score on this problem will
    be 25}
    \underline{out of 30.}
\item \underline{Bonus version:} You must find the quality of a maximum-quality
  path while enforcing both the minimum required number of turns and
  applying the turn penalty for each extra turn over the minimum. You
  will use all numbers in the input file. \underline{If you
    choose this version, your maximum possible score on this problem will
    be 35}
  \underline{out of 30.}
\end{enumerate}

For full credit, your solution must:
\begin{itemize}
\item Output a single path, followed by a single whitespace character, a single
  number (i.e., the total quality of an optimal path), and a newline
  character (so that each test output appears on its own line). The
  required format for the path, i.e., a  sequence of UPPERCASE letters
  (either `L' or `R'), is illustrated below.
\item Output the correct answer in the correct format for each and
  every test that I will run.
\item Run as fast as possible.
\item Must contain efficient code that uses built-in libraries to
  the full extent, and minimizes wasted time and space consumption.
\end{itemize}

I will run my tests in the following order:
\begin{enumerate}
\item test suite for the simplified version
\item test suite  for the basic version
\item test suite  for the bonus version
\end{enumerate}

I will stop my testing at the end of the first test suite in which
your submission fails at least one test. This will determine your
maximum possible score for this problem.  Your actual score will be
determined based on how close you were to passing all tests.

For reference, here is the expected output for inputs that use the
graph depicted above with different values of the first two parameters:

\begin{enumerate}
\item Simplified version:
  \begin{itemize}
  \item Input: \verb+0 0 2 10 15 10 8 4 8 5 17 8 3 10 7+
  \item Output: \verb+RLLR 45+
  \end{itemize}
\item Basic version:
  \begin{itemize}
  \item Input1: \verb+0 3 2 10 15 10 8 4 8 5 17 8 3 10 7+
  \item Output1: \verb+RLLR 39+
  \item Input2: \verb+0 6 2 10 15 10 8 4 8 5 17 8 3 10 7+
  \item Output2: \verb+RRLL 34+
  \end{itemize}
\item Bonus version:
  \begin{itemize}
  \item Input1: \verb+3 0 2 10 15 10 8 4 8 5 17 8 3 10 7+
  \item Output1: \verb+RLRL 42+
  \item Input2: \verb+3 1000 2 10 15 10 8 4 8 5 17 8 3 10 7+
  \item Output2: \verb+RLRL 42+
  \item Input3: \verb+2 0 2 10 15 10 8 4 8 5 17 8 3 10 7+
  \item Output3: \verb+RLLR 45+
  \item Input4: \verb+2 6 2 10 15 10 8 4 8 5 17 8 3 10 7+
  \item Output4: \verb+RLLR 45+
  \item Input5: \verb+1 6 2 10 15 10 8 4 8 5 17 8 3 10 7+
  \item Output5: \verb+RRLL 40+
  \item Input6: \verb+0 1 2 10 15 10 8 4 8 5 17 8 3 10 7+
  \item Output6: \verb+RLLR 43+
  \end{itemize}
\end{enumerate}

Your answer to this problem will be fully contained in your submitted
{\tt A10.java} file. For full credit, you must follow the directions given
therein.
}
\end{enumerate}


\end{document}
